%\documentclass{article}
%\begin{document}
\subsection{Master theses\label{exjobb}}
\begin{small}
\begin{enumerate}


%\item
%{\bf Thesis title}\\
%{\em Student:} \\
%{\em Supervisors:} \\
%{\em Subject supervisor:} \\
%{\em Publisher:} CBA Master Thesis No.~XXX / UPTEC FXX XXX\\
%{\em Abstract:} 

\item\textbf{An applied model for implementation of innovative IT-solutions for telehealth into the healthcare system}\\
  \emph{Student:} Hannah Lundstr\"{o}m, Tomas Berglund and Sara Lycke\\
  \emph{Supervisor:} Per Matsson, Cenvigo AB, Uppsala, Anders Hast \\ 
  \emph{Reviewer:} Tomas Nyberg, Dept.~of Engineering Sciences, and G\"{o}ran Lindstr\"{o}m, Dept.~of Engineering Sciences\\
  \emph{Publisher:} UPTEC F 14019\\
  \emph{Abstract:} Today, new technologies are introduced to the market every day, and constantly changing our way of living. Especially in the healthcare sector, the change process is approaching a point where doctors can benefit from the use of, for example, connected portable reading devices instead of paper-based medical record systems. The information and communication technology is promoting the evolution of a new pathway of care delivery, a paradigm shift that alters the fundamental relationship between a doctor and its patient. The concept is defined as telehealth and formulates the provision of care at a distance and provides the possibility to treat patients in their home environment instead of at the hospital.

This master's thesis has been performed on the request of Cenvigo AB, a company active in the implementation of new IT-solutions into the healthcare and eldercare. Cenvigo AB are the owners of the Parkinson's Digital Assessment (PANDA) application. The application has been developed through research at Dalarna University and Uppsala University Hospital.  This project will initiate the launch of PANDA and also create a model for implementation of innovative IT-solutions into the healthcare systems.

The model is founded in a theoretical framework and shaped with interviews related to the implementation of technology with a focus on telehealth applications. Interviews has been performed with healthcare professionals, technology developers and users to acquire a complete picture and opinions regarding the introduction of innovation in healthcare today. From the acquired information, a model is formulated as a stepwise and chronological linear process were identified key activities are included to promote a successful implementation process. The model is connected to the practice through the implementation of PANDA. In the process of implementing PANDA into the Swedish healthcare system, a collaboration with the innovation centre at Uppsala University Hospital as a healthcare organization stakeholder, has been initiated.

The model consists of five phases; Assessment, Dissemination, Adoption, Implementation and Continuation. The phases are seen as transitional steps in the innovation process, critical barriers to overcome towards a successfully implementation in a mainstream routine setting. Each phase includes a number of activities and to achieve progression in each phase, these activities must be performed in order to advance to the next phase. In the case of PANDA, the process of progression has passed assessment and are currently involved in activities related to the dissemination phase.

The purpose of the model is to be used both for existing and future applications in the segment of medicine technology sector. The structure of the model is designed to promote a co-design or a common value principle of development and practice regarding an innovation. By connecting actors from both technology and healthcare in close relationships the actual needs of healthcare professionals could more effectively be identified and developed into a solution, a result from the amplification of a two-way engagement. The outmost aim is to serve as a catalysing factor, complementing the implementation models of healthcare in Sweden today.

Through this study, a need for facilitating the implementation process of new technology into the healthcare systems has been identified. This model offers the necessary input that many technology companies lack. The recommendation to Cenvigo AB is to continue to develop the model during the last step in the process of launching PANDA, and parallel use this model as a business model mainly for technology start-ups and larger foreign companies that has not yet established pathways into the Swedish healthcare system.

\item\textbf{Adobe Flash Professional for iOS Game Development : A Feasible and Viable Alternative to Xcode?}\\
  \emph{Student:} Leila Svantro\\
  \emph{Supervisor:} Christopher Okhravi, Dept.~of Informatics and Media,  Faculty of Social Sciences, UU\\
  \emph{Reviewer:} Anders Hast and Olle G\"{a}llmo, Dept.~of IT, UU\\
  \emph{Publisher:} UPTEC IT, 14 028\\
  \emph{Abstract:} The smartphone operating system iOS is the second highest ranked after Android. The apps in App Store and Google Play combined consist of 70-80 \% games, which are the primary entertainment applications. Many developers are learning game development or refreshing their skills to profit on this trend. The problem statements are: is it viable and feasible to use Adobe Flash Professional (AFP) for the iOS game development compared to Xcode and could AFP be used exclusively for iOS game development? Information on both IDEs has been analyzed. Furthermore, implementations and code comparisons have been made. The results and analysis shows differences regarding expenses while possibilities for developing the same kind of games essentially are equivalent. The conclusions are that AFP is a viable IDE for iOS game development in the aspect of possibilities. It is not feasible on a long-term basis when considering the expenses however it could be feasible on a short-term basis depending on the developer's requirements of extension and Mac OS for App Store publishing. AFP is not able to be used exclusively for the iOS game development if publishing  to the App Store is a requirement however it is if publishing  is restricted to single devices.

\item\textbf{3D rendering and interaction in an augmented reality mobile system}\\
  \emph{Student:} Gabriel Tholsg{\aa}rd\\
  \emph{Supervisor:} Aman Hamdan, BMW Group, Shanghai, China \\
  \emph{Reviewer:} Anders Hast and Olle G\"{a}llmo, Dept.~of IT, UU\\
  \emph{Publisher:} UPTEC IT, 14 007\\
  \emph{Abstract:} Augmented Reality(AR) is a concept that is getting more and more popular, and the number of applications using it is increasing. AR applications include several concepts such as image recognition and camera calibration, together known as tracking, and it also uses 2D and 3D graphics rendering. The most important and difficult part of AR applications is the tracking, where an object not only should be recognized in many different conditions, but it should also be determined how the object is viewed upon. This report describes how the task given by BMW Group in Shanghai was solved, which was to create an iPhone prototype AR application, that should be able to recognize objects inside of a car and be able to interact with them through the mobile phone. The report explains the implemented solution to this problem, what different recognition methods were tested and the different ways of creating the 3D graphics overlay that was evaluated. The AR application resulted in a functional AR application capable of recognizing the determined objects, draw their corresponding 3D representations and interact with them. However, the application was not complete as camera calibration was not used and a connection between the mobile phone andthe car was never established.

\item\textbf{Realtime Virtual 3D Image of Kidney Using Pre-Operative CT Image for Geometry and Realtime US-Image for Tracking}\\
  \emph{Student:} Sebastian \"{A}rleryd\\
  \emph{Supervisor:} Massimiliano Collarieti-Tosti, KTH, Stockholm\\ 
  \emph{Reviewer:} Anders Hast and Anders Nyberg, Dept.~of T, UU\\
  \emph{Publisher:} UPTEC F 14042\\
  \emph{Abstract:} In this thesis a method is presented to provide a 3D visualization of the human kidney and surrounding tissue during kidney surgery. The method takes advantage of the high detail of 3D X-Ray Computed Tomography (CT) and the high time resolution of Ultrasonography (US). By extracting the geometry from a single preoperative CT scan and animating the kidney by tracking its position in real time US images, a 3D visualization of the surgical volume can be created. The first part of the project consisted of building an imaging phantom as a simplified model of the human body around the kidney. It consists of three parts: the shell part representing surrounding tissue, the kidney part representing the kidney soft tissue and a kidney stone part embedded in the kidney part. The shell and soft tissue kidney parts was cast with a mixture of the synthetic polymer Polyvinyl Alchohol (PVA) and water. The kidney stone part was cast with epoxy glue. All three parts where designed to look like human tissue in CT and US images. The method is a pipeline of stages that starts with acquiring the CT image as a 3D matrix of intensity values. This matrix is then segmented, resulting in separate polygonal 3D models for the three phantom parts. A scan of the model is then performed using US, producing a sequence of US images. A computer program extracts easily recognizable image feature points from the images in the sequence. Knowing the spatial position and orientation of a new US image in which these features can be found again allows the position of the kidney to be calculated. The presented method is realized as a proof of concept implementation of the pipeline. The implementation displays an interactive visualization where the kidney is positioned according to a user-selected US image scanned for image features. Using the proof of concept implementation as a guide, the accuracy of the proposed method is estimated to be bounded by the acquired image data. For high resolution CT and US images, the accuracy can be in the order of a few millimeters. 

\item\textbf{Detection and Quantification of Small Changes in MRI Volumes}\\
  \emph{Student:} Mariana Bustamante\\
  \emph{Supervisor:} Robin Strand\\
  \emph{Reviewer:} Anders Brun and Christoff Ivan, Dept.~of IT\\
  \emph{Publisher:} IT, 14 013\\
  \emph{Abstract:} The focus of this research is to attempt to solve the problem of comparing two MRI brain volumes of the same subject taken at different times, and detect the location and size of the differences between them, especially when such differences are too small to be perceived with the naked eye.

The research focuses on a combination of registration and morphometry techniques in order to create two different possible solutions: A voxel-based method and a tensor-based method. The first method uses Affine or B-Spline registration combined with voxel-by-voxel subtraction of the volumes; the second method uses Demons registration and analysis of the Jacobian determinants at each point of the deformation field obtained. The methods are implemented as modules for 3D Slicer, a software for medical image analysis and visualization.

Both methods are tested on two types of experiments: Artificial experiments, in which made-up differences of distinct sizes are added to volumes of healthy subjects; and real experiments, in which MRIs of real patients are compared.

The results obtained from the voxel-based method are very useful, since it was able to detect with almost complete accuracy all of the artificial differences and expected real differences during the experiments.

The tensor-based method's results are not as accurate in location or size of the detected  differences, and it usually includes more areas of differences where there seems to be none; even though it behaves adequately when the differences are large.

Most of the results obtained are useful for the diagnostic of patients with non-severe trauma to the head; especially when using the voxel-based method. However, the results from both methods are just a suggestion of the size and location of injuries; and as a consequence, the procedure  requires the presence of a medical practitioner.

\item\textbf{A Method for Detecting Resident Space Objects and Orbit Determination Based on Star Trackers and Image Analysis}\\
  \emph{Student:} Karl Bengtsson Bernander\\
  \emph{Supervisor:} Daniel Skaborn, {\AA}AC Microtec, Uppsala\\
  \emph{Reviewer:} Cris Luengo and Tomas Nyberg, Dept.~of IT\\
  \emph{Publisher:} UPTEC F 14050 \\
  \emph{Abstract:} Satellites commonly use onboard digital cameras, called star trackers. A star tracker determines the satellite's attitude, i.e. its orientation in space, by comparing star positions with databases of star patterns. In this thesis, I investigate the possibility of extending the functionality of star trackers to also detect the presence of resident space objects (RSO) orbiting the earth. RSO consist of both active satellites and orbital debris, such as inactive satellites, spent rocket stages and particles of different sizes.

I implement and compare nine detection algorithms based on image analysis. The input is two hundred synthetic images, consisting of a portion of the night sky with added random Gaussian and banding noise. RSO, visible as faint lines in random positions, are added to half of the images. The algorithms are evaluated with respect to sensitivity (the true positive rate) and specificity (the true negative rate). Also, a difficulty metric encompassing execution times and computational complexity is used.

The Laplacian of Gaussian algorithm outperforms the rest, with a sensitivity of 0.99, a specificity of 1 and a low difficulty. It is further tested to determine how its performance changes when varying parameters such as line length and noise strength. For high sensitivity, there is a lower limit in how faint the line can appear.

Finally, I show that it is possible to use the extracted information to roughly estimate the orbit of the RSO. This can be accomplished using the Gaussian angles-only method. Three angular measurements of the RSO positions are needed, in addition to the times and the positions of the observer satellite. A computer architecture capable of image processing is needed for an onboard implementation of the method.

\item\textbf{A Study of Digital In-Line Holographic Microscopy for Malaria Detection}\\
  \emph{Student:}  Carl Christian Kirchmann, Elin Lundin and Jakob Andr\'{e}n \\
  \emph{Supervisor:} Cris Luengo\\
  \emph{Reviewer:} Martin Sj\"{o}din, Dept.~of Engineering Sciences\\
  \emph{Publisher:} UPTEC TVE 14 054\\
  \emph{Abstract:} The main purpose of the project was to create an initial lab set-up for a dig-ital in-line holographic microscope and a reconstruction algorithm. Different parameters including: light source, pin-hole size and distances pinhole-object and object-camera had to be optimized. The lab set-up is to be developed further by a master student at the University of Nairobi and then be used for malaria detection in blood samples. To acquire good enough resolution for malaria detection it has been found necessary to purchase a gray scale camera with smaller pixel size. Two dierent approaches, in this report called the on-sensor approach and the object-magnication approach, were investigated. A reconstruction algorithm anda phase recovery algorithm was implemented as well as a super resolution algorithm to improve resolution of the holograms. The on-sensor approach proved easier and cheaper to use with approximately the same results as the object-magnication method. Necessary further research and development of experimental set-up was thoroughly discussed.

\item\textbf{Moving Object Detection based on Background Modeling}\\
  \emph{Student:} Yuanqing  Luo\\
  \emph{Supervisor:} Changqing Yin, Tongji University, Shanghai, China \\
  \emph{Reviewer:} Cris Luengo and Christoff Ivan, Dept.~of IT\\
  \emph{Publisher:} UPTEC IT 14 046 \\
  \emph{Abstract:} Aim at the moving objects detection, after studying several categories of background modeling methods, we design an improved Vibe algorithm based on image segmentation algorithm. Vibe algorithm builds background model via storing a sample set for each pixel. In order to detect moving objects, it uses several techniques such as fast initialization, random update and classification based on distance between pixel value and its sample set. In our improved algorithm, firstly we use histograms of multiple layers to extract moving objects in block-level in pre-process stage. Secondly we segment the blocks of moving objects via image segmentation algorithm. Then the algorithm constructs region-level information for the moving objects, designs the classification principles for regions and the modification mechanism among neighboring regions. In addition, to solve the problem that the original Vibe algorithm can easily introduce the ghost region into the background model, the improved algorithm designs and implements the fast ghost elimination algorithm. Compared with the tradition pixel-level background modeling methods, the improved method has better  robustness and reliability against the factors like background disturbance, noise and existence of moving objects in the initial stage. Specifically, our algorithm improves the precision rate from 83.17\% in the original Vibe algorithm to 95.35\%, and recall rate from 81.48\% to 90.25\%.

Considering the affection of shadow to moving objects detection, this paper designs a shadow elimination algorithm based on Red Green and Illumination (RGI) color feature, which can be converted from RGB color space, and dynamic match threshold. The results of experiments demonstrate  that the algorithm can effectively reduce the influence of shadow on the moving objects detection.

At last this paper makes a conclusion for the work of this thesis and discusses the future work.

\item\textbf{Pixel-based video coding}\\
  \emph{Student:} Johannes Olsson Sandgren\\
  \emph{Supervisor:} Jonatan Samuelsson, Ericsson AB.\\
  \emph{Reviewer:} Cris Luengo and Lars-{\AA}ke Nord\'{e}n, Dept.~of IT\\
  \emph{Publisher:} UPTEC IT 14 003\\
  \emph{Abstract:} This paper studies the possibilities of extending the pixel-based compression algorithm LOCO-I, used by the lossless and near lossless image compression standard JPEG-LS, introduced by the Joint Photographic Experts Group (JPEG) in 1999, to video sequences and very low bit-rates. Bitrates below 1 bit per pixel are achieved through skipping signaling when the prediction of a pixels sufficiently good. The pixels to be skipped are implicitly detected  by the decoder, minimizing the overhead. Different methods of quantization are tested, and the possibility of using vector quantization is investigated, by matching pixel sequences against a dynamically generated vector tree. Several different prediction schemes are evaluated, both linear and non-linear, with both static and adaptive weights.  Maintaining the low computational complexity of LOCO-I has been a priority. The results are compared to different HEVC implementations with regards to compression speed and ratio.

\item\textbf{Usability Analysis of SmartPaint}\\
  \emph{Student:} Nadia R\"{o}ning \\
  \emph{Supervisor:} Filip Malmberg\\
  \emph{Reviewer:} Mats Lind and Olle G\"{a}llmo, Dept.~of IT, UU\\
  \emph{Publisher:} UPTEC IT 14 051\\
  \emph{Abstract:} Image segmentation is the process of identifying and separating relevant objects and structures in an image. The purpose of segmentation is to simplify and/or change the representation of an image into something that is easier to analyze. SmartPaint is a software for semi-automatic segmentation of medical volume  images, developed by Filip Malmberg.  This thesis investigates whether SmartPaint is useful on several levels, such as usability, functionality and instructional effectiveness. The developer's ambition is that SmartPaint should be accessible to users without a background in computer science. Hence a formative usability study (Cooperative evaluation) was conducted, involving testing and interviewing participants. Given  the result from the study and feedback from the participants, design proposals are given. Furthermore, ideas on how to expand the functionality, the instructional effectiveness and the learnability of SmartPaint are given.

\item\textbf{An interactive interface for multiple-resolution analysis of large images}\\
  \emph{Student:} Nguyen-Anh-Thu Tran\\
  \emph{Supervisor:} Petter Ranefall\\
  \emph{Reviewer:} Carolina W\"{a}hlby and Olle G\"{a}llmo, Dept.~of IT, UU\\
  \emph{Publisher:} UPTEC IT 14 034\\
  \emph{Abstract:} Digital image analysis has contributed greatly to medical sciences. Modern microscope slide scanning systems are capable of producing large images which can be more than one giga-pixel. It is useful for researchers to be able to view these images at multiple resolutions. For instance, to implement image-based sequencing of messenger ribonucleic acid (mRNA), high resolution images are required in detailed analysis while those at low resolution offer better  overall visualization. Taking that as the motivation, a map-viewer-like user interface with zooming and panning options has been developed to support detailed analysis in high resolution and at the same time be able to get a full overview in lower resolution. This thesis describes the context in which the interface is used as well as its design process.

\item\textbf{Methods for automatic analysis of glucose uptake in adipose tissue using quantitative PET/MRI data}\\
  \emph{Student:} Jonathan Andersson \\
  \emph{Supervisor:} Joel Kullberg, Dept.~of ROS, Faculty of Medicine, UU\\
  \emph{Reviewer:} Robin Strand and Tomas Nyberg, Dept.~of IT\\
  \emph{Publisher:} UPTEC F 14044\\
  \emph{Abstract:} Brown adipose tissue (BAT) is the main tissue involved in non-shivering heat production. A greater understanding of BAT could possibly lead to new ways of prevention and treatment of obesity and type 2 diabetes. The increasing prevalence of these conditions and the problems they cause society and individuals make the study of the subject important.

An ongoing study performed at the Turku University Hospital uses images acquired using PET/MRI with 18F-FDG as the tracer. Scans are performed on sedentary and athlete subjects during normal room temperature and during cold stimulation. Sedentary subjects then undergo scanning during cold stimulation again after a six weeks long exercise training intervention. This degree project used images from this study.

The objective of this degree project was to examine methods to automatically and objectively quantify parameters relevant for activation of BAT in combined PET/MRI data. A secondary goal was to create images showing glucose uptake changes in subjects from images taken at different times.

Parameters were quantified in adipose tissue directly without registration (image matching), and for neck scans also after registration. Results for the first three subjects who have completed the study are presented. Larger registration errors were encountered near moving organs and in regions with less information.

The creation of images showing changes in glucose uptake seem to be working well for the neck scans, and somewhat well for other sub-volumes. These images can be useful for identification of BAT. Examples of these images are shown in the report.\\
\emph{Comment: }In cooperation with Turku University Hospital

\item\textbf{Cell Tracking in Microscopy Images Using a Rao-Blackwellized Particle Filter}\\
  \emph{Student:} Sofia Lindmark\\
  \emph{Supervisor:} Thomas Sch\"{o}n, Dept.~of IT\\
  \emph{Reviewer:} Carolina W\"{a}hlby and Tomas Nyberg, Dept.~of IT \\
  \emph{Publisher:} UPTEC F 14048\\
  \emph{Abstract:} Analysing migrating cells in microscopy time-lapse images has already helped the understanding of many biological processes and may be of importance in the development of new medical treatments. Today's biological experiments tend to produce a huge amount of dynamic image data and tracking the individual cells by hand has become a bottleneck for the further analysis work. A number of cell tracking methods have therefore been developed over the past decades, but still many of the techniques have a limited performance.

The aim of this Master Project is to develop a particle filter algorithm that automatically detects and tracks a large number of individual cells in an image sequence. The solution is based on a Rao-Blackwellized particle filter for multiple object tracking. The report also covers a review of existing automatic cell tracking techniques, a review of well-known filter techniques for single target tracking and how these techniques have been developed to handle multiple target tracking. The designed algorithm has been tested on real microscopy image data of neutrophils with 400 to 500 cells in each frame. The designed algorithm works well in areas of the images where no cells touch and can in these situations also correct for some segmentation mistakes. In areas where cells touch, the algorithm works well if the segmentation is correct, but often makes mistakes when it is not. A target effectiveness of 77 percent and a track purity of 80 percent are then achieved. 
%
%\item\textbf{}\\
%  \emph{Student:} \\
%  \emph{Supervisor:} \\
%  \emph{Reviewer:} \\
%  \emph{Publisher:} \\
%  \emph{Abstract:} 
%



\end{enumerate}
\end{small}
%\end{document}