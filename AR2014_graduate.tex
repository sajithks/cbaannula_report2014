\section{Graduate education} 

{\large
We give a number of PhD courses each year, both for our own students and for PhD students in subjects that use image analysis as a tool and need to know more about it. This year Cris Luengo gave a new course on Scientific Data Presentation. The available places were filled within an hour and we expect to give this popular course soon again. Carlina W\"{a}hlby gave several courses to researchers in biomedicine, Ida-Maria Sintorn gave a course focussed on microscopy applications in Ume{\aa} and Robin Strand gave our long-running course in Application Oriented Image Analysis once again.

There were seven PhD defences this year. 

%Listed in this section are graduate couses we have been involved in, as course organizers and/or teachers.

%At the end of 2012, we were main supervisors for 16 PhD students, eleven at UU and four at SLU. 
%Borgefors is also assistant supervisor for a PhD student at the Dept.~of Biometry and Engineering, SLU.
%
}
\vspace*{-3mm}
\subsection{Graduate courses}
{\small
\begin{enumerate}
%%%%
%
\item
{\bf Research methodology for information technology, 4hp}~\\
{\em Examiner:} Gunilla Borgefors~\\
{\em Lecturer(s):} Gunilla Borgefors, Ulrika Haak~\\
{\em Period:} 140424--0630~\\
{\em Venue:} The course was given at CBA\\
{\em Description:} The goal is to give general and useful knowledge about how to become a good and published researcher in information technology and/or various applications thereof. The first  part consists of five traditional lectures on general themes. The  second  part is a series of seminars held by the participants, describing a relevant scientific conference or journal and making an oral and a written report.~\\
{\em Comment:} The written reports are collected in CBA internal Report No. 53.
%
\item
{\bf Scientific Data Presentation, 2hp}\\
Gunilla Borgefors, {\bf Cris Luengo}\\
{\em Period:} 141008--1105\\
{\em Venue:} The course was given at CBA\\
{\em Description:} The goal of the course is to give PhD students the ability to effectively present the data resulting from their experiments. The course covered different forms of graphs and tables for one and two- dimensional sampled data, categorical data, discrete values, etc.; certain aspects of human perception relevant to displaying data, including colour perception; the need to highlight the story in the data, refraining from displaying the non-essential things (without, of course, misrepresenting the data); and how to use drawing tools such as Illustrator or Inkscape to edit figures generated by Excel, MATLAB, or any other graphing tool. \\
 {\em Comment:} Funded by FUN, TekNat

\item
{\bf Live Cell Imaging, 3hp}~\\
{\bf Ida-Maria Sintorn}~\\
{\em Venue:} The course was held at KI in Huddinge.~\\
{\em Period:} 141013--1013
%

\item
{\bf Basic Image Analysis: Focused on Microscopy Applications, 3hp}~\\
Cris Luengo, {\bf Ida-Maria Sintorn}, Carolina W\"{a}hlby~\\
{\em Period:} 140917--1009\\
{\em Venue:} The course was held at KI in Huddinge.~\\
{\em Description:} This postgraduate course in Basic Image Analysis was organized administered by the Regenerative Medicine Doctoral Program for PhD students at Karolinska Institutet and KTH.\\
{\em Comment:} PhD level course: 9 lectures, 2 computer exercises (Fiji, CellProfiler), 1 week project on own data

\item
{\bf Classical \& Modern Papers}~\\
PhD students at CBA, {\bf Cris Luengo}~\\
{\em Period:} During the whole year\\
{\em Venue:} The course was given at CBA\\
{\em Description:} Presentations and discussions of classical or modern papes in image processing.~

\item
{\bf Scientific Visualization Workshop, 1hp}~\\
{\bf Anders Hast}, Johan Nysj\"{o}, Pontus Olsson~\\
{\em Period:} 141127--1128\\
{\em Venue:} The workshop was given at CBA\\
{\em Description:} A workshop for PhDs and other researchers arranged by SNIC-UPPMAX and SeSE.
%



%%%%%
\end{enumerate}
}


%%%%%%%%%%%%%%%%%%%%%%%%%%%%%%%%%%%%
%%%%%%%%%%%%%%%%%%%%%%%%%%%%%%%%%%%%
\newpage
\subsection{Dissertations}\label{phd}
{\small
\begin{enumerate}
%\item
%%%%%%%%%%%%%%%%%%%%%%%%%%%%%%%%%%%%%%%%%%%%%%%%%%%%%%%%%%%%%%%%%%
%\end{enumerate}


\item
{\em Date:} 141020~\\
{\bf Automated Tissue Image Analysis Using Pattern Recognition}~\\
{\em Student:} Jimmy Azar~\\
{\em Supervisor:} Anders Hast~\\
{\em Assistant Supervisor:} Ewert Bengtsson and Martin Simonsson~\\
{\em Opponent:} Marco Loog, Pattern Recognition \& Bioinformatics Group, Delft University of Technology, The Netherlands~\\
{\em Committee:}  Gunilla Borgefors, Thomas Sch\"{o}n, Mats Gustafsson, Department of Information Technology, Uppsala University; Arne \"{O}stman, Department of Oncology-Pathology, Karolinska Institutet; Fritz Albregtsen, Department of Informatics, University of Oslo~\\
{\em Publisher:} Acta Universitatis Upsaliensis, ISBN: 978-91-554-9028-7 ~\\ 
{\em Abstract:} Automated tissue image analysis aims to develop algorithms for a variety of histological applications. This has important implications in the diagnostic grading of cancer such as in breast and prostate tissue, as well as in the quantification of prognostic and predictive biomarkers that may help assess the risk of recurrence and the responsiveness of tumors to endocrine therapy.
In this thesis, we use pattern recognition and image analysis techniques to solve several problems relating to histopathology and immunohistochemistry applications. In particular, we present a new method for the detection and localization of tissue microarray cores in an automated manner and compare it against conventional approaches.
We also present an unsupervised method for color decomposition based on modeling the image formation process while taking into account acquisition noise. The method is unsupervised and is able to overcome the limitation of specifying absorption spectra for the stains that require separation. This is done by estimating reference colors through fitting a Gaussian mixture model trained using expectation-maximization.
Another important factor in histopathology is the choice of stain, though it often goes unnoticed. Stain color combinations determine the extent of overlap between chromaticity clusters in color space, and this intrinsic overlap sets a main limitation on the performance of classification methods, regardless of their nature or complexity. In this thesis, we present a framework for optimizing the selection of histological stains in a manner that is aligned with the final objective of automation, rather than visual analysis.
Immunohistochemistry can facilitate the quantification of biomarkers such as estrogen, progesterone, and the human epidermal growth factor 2 receptors, in addition to Ki-67 proteins that are associated with cell growth and proliferation. As an application, we propose a method for the identification of paired antibodies based on correlating probability maps of immunostaining patterns across adjacent tissue sections.
Finally, we present a new feature descriptor for characterizing glandular structure and tissue architecture, which form an important component of Gleason and tubule-based Elston grading. The method is based on defining shape-preserving, neighborhood annuli around lumen regions
and gathering quantitative and spatial data concerning the various tissue-types.



\item
{\em Date:} 20140411~\\
{\bf Image Analysis Methods and Tools for Digital Histopathology Applications Relevant to Breast Cancer Diagnosis}~\\
{\em Student:} Andreas K{\aa}rsn\"{a}s~\\
{\em Supervisor:} Robin Strand~\\
{\em Assistant Supervisor:} Ewert Bengtsson and Carolina W\"{a}hlby~\\
{\em Opponent:} Anant Madabhushi, Case Western Reserve University, OH, US ~\\
{\em Committee:} Johan Lundin, Institute for Molecular Medicine Finland, Helsinki, Finland; Arne \"{O}stman, Karolinska Insitute, Stockholm; Andrew Mehnert, Chalmers Univ. of Technology; Irene Yu-Hua Gu, Chalmers Univ. of Technology; Anders Hast ~\\
{\em Publisher:} Acta Universitatis Upsaliensis, ISBN: 978-91-554-8889-5 ~\\ 
{\em Abstract:} In 2012, more than 1.6 million new cases of breast cancer were diagnosed and about half a million women died of breast cancer. The incidence has increased in the developing world. The mortality, however, has decreased. This is thought to partly be the result of advances in diagnosis and treatment. Studying tissue samples from biopsies through a microscope is an important part of diagnosing breast cancer. Recent techniques include camera-equipped microscopes and whole slide scanning systems that allow for digital high-throughput scanning of tissue samples. The introduction of digital pathology has simplified parts of the analysis, but manual interpretation of tissue slides is still labor intensive and costly, and involves the risk for human errors and inconsistency. Digital image analysis has been proposed as an alternative approach that can assist the pathologist in making an accurate diagnosis by providing additional automatic, fast and reproducible analyses. This thesis addresses the automation of conventional analyses of tissue, stained for biomarkers specific for the diagnosis of breast cancer, with the purpose of complementing the role of the pathologist. In order to quantify biomarker expression, extraction and classification of sub-cellular structures are needed. This thesis presents a method that allows for robust and fast segmentation of cell nuclei meeting the need for methods that are accurate despite large biological variations and variations in staining. The method is inspired by sparse coding and is based on dictionaries of local image patches. It is implemented in a tool for quantifying biomarker expression of various sub-cellular structures in whole slide images. Also presented are two methods for classifying the sub-cellular localization of staining patterns, in an attempt to automate the validation of antibody specificity, an important task within the process of antibody generation.  In addition, this thesis explores methods for evaluation of multimodal data. Algorithms for registering consecutive tissue sections stained for different biomarkers are evaluated, both in terms of registration accuracy and deformation of local structures. A novel region-growing segmentation method for multimodal data is also presented. In conclusion, this thesis presents computerized image analysis methods and tools of potential value for digital pathology applications.

\item
{\em Date:} ~\\
{\bf }Automatic Virus Identification using TEM -- Image Segmentation and Texture Analysis~\\
{\em Student:} Gustaf Kylberg~\\
{\em Supervisor:} Ida-Maria Sintorn ~\\
{\em Assistant Supervisor:} Gunilla Borgefors~\\
{\em Opponent:} Walter Kropatsch, Vienna University of Technology~\\
{\em Committee:}  Stina Svensson, Ray Search Labs, Stockholm; Magnus Borga, Link\"{o}ping University; Robin Strand; Abdenour Hadid, Oulo University, Finland; Kjell Hultenby, Karolinska Institute, Stockholm~\\
{\em Publisher:} Acta Universitatis Upsaliensis, ISBN: 978-91-554-8873-4~\\ 
{\em Abstract:} Viruses and their morphology have been detected and studied with electron microscopy (EM) since the end of the 1930s. The technique has been vital for the discovery of new viruses and in establishing the virus taxonomy. Today, electron microscopy is an important technique in clinical diagnostics. It both serves as a routine diagnostic technique as well as an essential tool for detecting infectious agents in new and unusual disease outbreaks.

The technique does not depend on virus specific targets and can therefore detect any virus present in the sample. New or reemerging viruses can be detected in EM images while being unrecognizable by molecular methods.

One problem with diagnostic EM is its high dependency on experts performing the analysis. Another problematic circumstance is that the EM facilities capable of handling the most dangerous pathogens are few, and decreasing in number.

This thesis addresses these shortcomings with diagnostic EM by proposing image analysis methods mimicking the actions of an expert operating the microscope. The methods cover strategies for automatic image acquisition, segmentation of possible virus particles, as well as methods for extracting characteristic properties from the particles enabling virus identification.

One discriminative property of viruses is their surface morphology or texture in the EM images. Describing texture in digital images is an important part of this thesis. Viruses show up in an arbitrary orientation in the TEM images, making rotation invariant texture description important. Rotation invariance and noise robustness are evaluated for several texture descriptors in the thesis. Three new texture datasets are introduced to facilitate these evaluations. Invariant features and generalization performance in texture recognition are also addressed in a more general context.

The work presented in this thesis has been part of the project Panvirshield, aiming for an automatic diagnostic system for viral pathogens using EM. The work is also part of the miniTEM project where a new desktop low-voltage electron microscope is developed with the aspiration to become an easy to use system reaching high levels of automation for clinical tissue sections, viruses and other nano-sized particles.

\item
{\em Date:} 20141205 ~\\
{\bf Characterisation of wood-fibre--based materials using image analysis}~\\
{\em Student:} Erik Wernersson~\\
{\em Supervisor:} Gunilla Borgefors~\\
{\em Assistant Supervisor:} Cris Luengo and Anders Brun~\\
{\em Opponent:} Michal Kozubek, Masaryk University, Brno, Czech Republic~\\
{\em Committee:}  Gunnar Sparr,Lund Institute of Technolog /Lund University; Bj\"{o}rn Kruse Link\"{o}ping University, \"{O}rjan Smedby Link\"{o}ping University~\\
{\em Publisher:}  Acta Universitatis agriculturae Sueciae, ISBN: 978-91-576-8146-1 ~\\ 
{\em Abstract:} Wood fibres are the main constituent of paper and are also used to alter properties of plastics in wood-fibre--based composite materials. The manufacturing of these materials involves numerous parameters that determine the quality of the products. The link between the manufacturing parameters and the final products can often be found in properties of the microstructure, which calls for advanced characterisation methods of the materials. Computerised image analysis is the discipline of using computers to automatically extract information from digital images. Computerised image analysis can be used to create automated methods suitable for the analysis of large data volumes. Inherently these methods give reproducible results and are not biased by individual analysts. In this thesis, three-dimensional X-ray computed tomography (CT) at micrometre resolution is used to image paper and composites. Image analysis methods are developed to characterise properties of individual fibres, properties of fibre--fibre bonds, and properties of the whole fibre networks based on these CT images. The main contributions of this thesis is the development of new automated image-analysis methods for characterisation of wood-fibre--based materials. This include the areas of fibre--fibre contacts and the free--fibre lengths. A method for reduction of phase contrast in mixed mode CT images is presented. This method retrieves absorption from images with both absorption and phase contrast. Curvature calculations in volumetric images are discussed and a new method is proposed that is suitable for three-dimensional images of materials with wood fibres, where the surfaces of the objects are close together.

\item
{\em Date:} ~\\
{\bf Image Analysis and Interactive Visualization Techniques for Electron Microscopy Tomograms}~\\
{\em Student:} Lennart Svensson~\\
{\em Supervisor:} Ida-Maria Sintorn~\\
{\em Assistant Supervisor:} Ingela Nystr\"{o}m, Gunilla Borgefors~\\
{\em Opponent:} Willy Wriggers, Associate Professor, Weill Conell Medical College \& Researcher, D.E. Shaw Research New York, US~\\
{\em Committee:} 
Hans Hebert, Karolinska Institutet, Stockholm; 
Natasa Sladoje, 
University of Novi Sad, Serbia; 
Stefan Seipel, CBA
\\
{\em Publisher:} Acta Universitatis agriculturae Sueciae, ISBN: 978-91-576-8136-2 ~\\ 
{\em Abstract:} Images are an important data source in modern science and engineering. A continued challenge is to perform measurements on and extract useful information from the image data, i.e., to perform image analysis. Additionally, the image analysis results need to be visualized for best comprehension and to enable correct assessments. In this thesis, research is presented about digital image analysis and three-dimensional (3-D) visualization techniques for use with transmission electron microscopy (TEM) image data and in particular electron tomography, which provides 3-D reconstructions of the nano-structures. The electron tomograms are difficult to interpret because of, e.g., low signal-to-noise ratio, artefacts that stem from sample preparation and insufficient reconstruction information. Analysis is often performed by visual inspection or by registration, i.e., fitting, of molecular models to the image data. Setting up a visualization can however be tedious, and there may be large intra- and inter-user variation in how visualization parameters are set. Therefore, one topic studied in this thesis concerns automatic setup of the transfer function used in direct volume rendering of these tomograms. Results indicate that histogram and gradient based measures are useful in producing automatic and coherent visualizations. Furthermore, research has been conducted concerning registration of templates built using molecular models. Explorative visualization techniques are presented that can provide means of visualizing and navigating model parameter spaces. This gives a new type of visualization feedback to the biologist interpretating the TEM data. The introduced probabilistic template has an improved coverage of the molecular flexibility, by incorporating several conformations into a static model. Evaluation by cross-validation shows that the probabilistic template gives a higher correlation response than using a Protein Databank (PDB) devised model. The software ProViz (for Protein Visualization) is also introduced, where selected developed techniques have been incorporated and are demonstrated in practice.

\item
{\em Date:} 140523~\\
{\bf } Distance Functions and Their Use in Adaptive Mathematical Morphology~\\
{\em Student:} Vladimir Curic~\\
{\em Supervisor:} Gunilla Borgefors~\\
{\em Assistant Supervisor:} Cris Luengo\\
{\em Opponent:} Hugues Talbot, University Paris-Est - ESIEE, France~\\
{\em Committee:} Christer Kiselman, UU; 
Gabriella Sanniti di Baja, Istituto di Cibernetica, Napoli, Italy; 
Alexander Medveded, UU; 
Reiner Lenz, Linköping University; 
Anders Heiden, Lund University 
 ~\\
{\em Publisher:} Acta Universitatis Upsaliensis, ISBN: 978-91-554-8923-6~\\ 
{\em Abstract:}One of the main problems in image analysis is a comparison of different shapes in images. It is often desirable to determine the extent to which one shape differs from another. This is usually a difficult task because shapes vary in size, length, contrast, texture, orientation, etc. Shapes can be described using sets of points, crisp of fuzzy. Hence, distance functions between sets have been used for comparing different shapes.

Mathematical morphology is a non-linear theory related to the shape or morphology of features in the image, and morphological operators are defined by the interaction between an image and a small set called a structuring element. Although morphological operators have been extensively used to differentiate shapes by their size, it is not an easy task to differentiate shapes with respect to other features such as contrast or orientation. One approach for differentiation on these type of features is to use data-dependent structuring elements.

In this thesis, we investigate the usefulness of various distance functions for: (i) shape registration and recognition; and (ii) construction of adaptive structuring elements and functions.

We examine existing distance functions between sets, and propose a new one, called the Complement weighted sum of minimal distances, where the contribution of each point to the distance function is determined by the position of the point within the set. The usefulness of the new distance function is shown for different image registration and shape recognition problems. Furthermore, we extend the new distance function to fuzzy sets and show its applicability to classification of fuzzy objects.

We propose two different types of adaptive structuring elements from the salience map of the edge strength: (i) the shape of a structuring element is predefined, and its size is determined from the salience map; (ii) the shape and size of a structuring element are dependent on the salience map. Using this salience map, we also define adaptive structuring functions. We also present the applicability of adaptive mathematical morphology to image regularization. The connection between adaptive mathematical morphology and Lasry-Lions regularization of non-smooth functions provides an elegant tool for image regularization.

\item
{\em Date:} ~\\
{\bf }~\\
{\em Student:} Patrik Malm~\\
{\em Supervisor:} ~\\
{\em Assistant Supervisor:} ~\\
{\em Opponent:} ~\\
{\em Committee:}  ~\\
{\em Publisher:} Acta Universitatis Upsaliensis, ISBN: ~\\ 
{\em Abstract:}



%\subsection{Doctoral conferment ceremonies}
%{\small
%\begin{enumerate}
%\item
%{\em Title:} Promotor for the Faculty of Forest Science, SLU\\
%{\bf Gunilla Borgefors}\\
%{\em Date:} 101009\\
%{\em Description:} The promotor confers the ceremony where those who have defended their PhD theses during the year receive their PhD diplomas and laurels or hats. The promotor each year is the most senior professor at the faculty that has not yet been promotor. There are a number of tasks apart from the actual conferment ceremony that was held in Latin, for the first time in the history of the Faculty. 
%\end{enumerate}
%\subsection*{Promotors at doctoral conferment ceremonies from CBA}
%\begin{enumerate}
%\item
%Ewert Bengtsson, TN-Faculty, UU, 2009
%\item
%Gunilla Borgefors, S-Faculty, SLU, 2010

\end{enumerate}
%}
}