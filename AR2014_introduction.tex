%%%%%%%%%%%%%%%%%%%%%%%%%%%%%%%%%%%%%%%%%%%%%%
\section{Introduction}
\noindent 
The Centre for Image Analysis (CBA) carries out research and graduate education in computerised image analysis and perceptualisation. Our work ranges from the pure theory to methods, algorithms and systems for applications primarily in biomedicine and forest industry.

\subsection{General background}\label{sect:background}

\noindent 
CBA is collaboration between Uppsala University (UU) and the Swedish University of Agricultural Sciences (SLU), which started in 1988. This means that CBA celebrated 25 years in 2013! From an organizational point of view, CBA was an independent entity within our host universities until 2010.

At UU, we are hosted by the Disciplinary Domain of Science and Technology and today belong to one of five divisions within the Dept.~of Information Technology (IT), the Division of Visual Information and Interaction (Vi2). At SLU, we today belong to the Dept.~of Forest Genetics and Plant Physiology in Ume{\aa}. The organizational matters are outlined in Section~\ref{sec:organization}. The re-organizations have not prevented us from continuing and expanding our research.
 We foresee opportunities for collaborations among our close colleagues at UU and SLU. 

During 2013, a total of 39 persons were working at CBA: 18 researchers, 19 PhD students, one technical staff, and one administrator. Additionally, 17 Master thesis students completed their thesis work with supervision from CBA. This does not mean, however, that we have had more than 50 full-time persons at CBA: many have split appointments, part time at CBA and part time elsewhere, adding up to approximately 30 full-time employments. Having world class scientists visiting CBA and CBA staff visiting their groups, for longer or shorter periods, is an important ingredient of our activities. 

Most of us at CBA also undertake some undergraduate teaching. Previously this has been organised by other divisions, but with the organizational changes our new division now handles undergraduate education. 

%We are pleased that Robin Strand and Ida-Maria Sintorn qualified as Docents at UU bringing the total number of CBA docents to twelve.

We can conclude that the activities remain high. On average, three PhD dissertations are produced each year at CBA. Nevertheless, in 2013 there was no PhD exam. On the other hand, we expect as many as eight (8!) PhD theses to be defended in 2014. In 2013, we published 50 internationally reviewed papers, more than any year before in the
history of CBA. There are several reasons for this. The main reason is that so many of our PhD students are at the end of their studies, which is when they publish most. Another reason is that we have more researchers than before and are involved in more co-operation projects.

We had continued support from the Disciplinary Domain of Medicine and Pharmacy, the Science for Life Laboratory (SciLifeLab), and strategic resources within the Dept.~of IT. The strong economy has led to recruitments of new PhD students and researchers during the year. A successful example of collaboration we have is with the Dept.~of Radiology, Oncology, and Radiation Sciences; Section of Radiology, where two of our staff members work part time in order to be close to radiology researchers. 

%In 2013, we have established ourselves within the scope for automatically reading old handwritings, so called HTR
In 2013, we have established ourselves within the field of automatic reading of old hand-written documents, referred to as HTR (Hand-written Text Recognition). The framework project is funded by VR, with support from the Vice Chancellor, and is truly multi-disciplinary, with partners from the Humanities and Social Sciences, and the Uppsala University Library.
%The VR-funded framework project is multi-disciplinary within UU with partners from among others Humanities and Social Sciences and the Uppsala University Library, and with support from the Vice Chancellor. This project represents a significant leap into the emerging field of digital humanities. 

An outreach activity that was particularly important was the 11th International Symposium on Mathematical Morphology (ISMM 2013) held in Uppsala in May with 69 participants. See \url{http://www.cb.uu.se/ismm2013}. Researchers from both universities were active in the arrangements.

Another outreach activity we have is our participation in the annual symposium on image analysis, arranged by the Swedish Society for Automated Image Analysis in March. In 2013, it was held in Gothenburg and CBA accounted for about a quarter of the participants with 20 registrations.
\clearpage
Image processing is highly inter- and multi-disciplinary, with foundations in mathematics, statistics, physics, signal processing and computer science, and with applications in many diverse fields. We are working in a wide range of application areas, most of them related to life sciences and usually in close collaboration with domain experts. Our collaborators are found locally as well as nationally and internationally. 
For a complete list of our 45 national and 30 international collaborators see Section~\ref{partners}. 

Ingela Nystr\"{o}m, our director, continues to coordinate the strategic research programme in the e-science field, eSSENCE. She terminated her position on the board of the Swedish University Computer Network, SUNET, during 2013. 

We are very active in international and national societies. 
Both Ewert Bengtsson and Gunilla Borgefors are elected members of the Royal Society of Sciences in Uppsala and the Royal Swedish Academy of Engineering Sciences (IVA). Ingela Nystr\"{o}m is elected member of the Royal Society of Arts and Sciences of Uppsala. 
Gunilla Borgefors is Editor in Chief for the journal Pattern Recognition Letters and Cris Luengo is Area Editor for the same journal. Ewert Bengtsson is associate editor of Computer Methods and Programs in Biomedicine. Ingela Nystr\"{o}m serves as Secretary of the International Association of Pattern Recognition, IAPR.
 Researchers at CBA also served on several other journal editorial boards, scientific organization boards, conference committees, and PhD dissertation committees. In addition, we took a very active part in reviewing grant applications and scientific papers submitted to conferences and journals. 

In addition to the more common ways of spreading information about our activities and work, such as seminar series, publications, web-pages, etc., we have our ``CBA TV''. Short ``trailers'' on our projects and activities are presented on an LCD monitor facing the main entrance stairway where students and colleagues from other groups pass by.

This annual report is also available on the CBA webpage, see \url{http://www.cb.uu.se/annual_report/AR2013.pdf}
\vfill
%%%%%%%%%%%%%%%%%%%%%%%%%%%%%%%%%%%%%%%%
%%%%%%%%%%%%%%%%%%%%%%%%%%%%%%%%%%%%%%%%
\subsection{Summary of research}\label{sect:summary}
The objective of CBA is to carry out research and education in computerised image analysis and perceptualisation. We are pursuing this objective through a large number of research projects, ranging from fundamental mathematical methods development, to application-tailored development and testing, the latter mainly in biomedicine and forest industry. We are also developing new methods for perceptualisation, combining computer graphics, haptics, and image processing.
Our research is organised in a large number of projects (53) of varying sise, ranging in effort from a few person months to many person years. There is a lot of interaction between different researchers: generally, a person is involved in several different projects in different constellations with internal and external partners. In this context, the university affiliation of the particular researchers seldom is of importance.

On the theoretical side, most of our work is based on discrete mathematics with fundamental work on sampling grids, fuzzy methods, skeletons, distance functions, and tessellations, in three and more dimensions.

Several projects deal with light microscopy, developing tools for modern quantitative biology and clinical cancer detection and grading. We are collaborating with local biologists and pathologists, research centers in the US and India, and a Danish company. We have close collaboration with the strategic project programme SciLifeLab through which a research platform in quantitative microscopy is formed. 

We also work with electron microscopy (EM) images; one application is focused on finding viruses in EM images. Since the texture of the virus particles is an important feature in identification of the different virus types, this project has also led to basic research on texture analysis. 
\clearpage
New techniques are creating 3D images on microscopic scales. We have been analyzing electron microscope tomography images of protein molecules for several years. We are also involved in optical projection tomography, where we image zebrafish embryos. Another technique is X-ray microtomography; we are developing methods to use such images to study the internal structure of paper, wood fibre composites and bone, and bone-implant integration. 

On a macroscopic scale, we are working with interactive segmentation of 3D CT and MR images by use of haptics. We have developed a segmentation toolbox, WISH, which is publicly available. Applications of this toolbox are for facial surgery planning and measurements of CT wrist images. 

Over the last several years, we have expanded our activities in perceptualisation under leadership of Guest Professor Ingrid Carlbom, with the goal of creating a system in which you can see, feel, and manipulate virtual 3D objects as if they were real. 
 We have created a unique haptic system where virtual objects can be grabbed and manipulated. This project has obvious synergy with the Human-Computer Interaction research performed within the Division Vi2. 

See Section~\ref{research} for details on all our research projects.

An activity bridging research and education is the supervision of master thesis projects. This year we completed 18 such projects. In Section~\ref{exjobb}, we describe these theses.
%\newpage

\subsection{How to contact CBA}

CBA maintains a home page (\url{http://www.cb.uu.se/}) both in English and in Swedish. The main structure contains links to a brief presentation, staff, vacant positions (if any), etc. It also contains information on courses, seminars (note that our Monday 14:15 seminar series is open to anyone interested), a layman introduction to image analysis, this annual report (as .html and .pdf versions), lists of all publications since CBA was created in 1988, and other material.

In addition, all staff members have their own home page, which are linked to from the CBA ``Staff'' page. On these, you can usually find detailed course and project information, etc.

Centre for Image Analysis (Centrum f\"{o}r bildanalys, CBA) can be contacted in the following ways:

\begin{tabular}{ll}
\\
\emph{Visiting address:\/} & L\"{a}gerhyddsv\"{a}gen 2\\
&Polacksbacken, building 2, floor 1\\
& Uppsala\\
~\\
\emph{Postal address:\/} & Box 337\\
& SE-751~05~~Uppsala\\
& Sweden\\
~\\
\emph{Telephone:\/} & +46 18 471 3460\\
\emph{Fax:\/} & +46 18 511925\\
\emph{E-mail:\/} & {\tt cb@cb.uu.se}
\end{tabular}
